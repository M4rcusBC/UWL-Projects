\documentclass{article}

\usepackage[margin=0.5in]{geometry}
\usepackage{amsfonts,xcolor}
\usepackage{amsmath, amsthm, enumitem} % The package amsmath will make it so that the \text{ words here } things work in the sets $S$ and $T$.

\newcommand\Z{\mathbb{Z}}
\newcommand\Q{\mathbb{Q}}
\newcommand\R{\mathbb{R}}
\newcommand\C{\mathbb{C}}

\begin{document}

\begin{enumerate}

\item 
Let $N = \{z \in \mathbb{Z} : z \not= 0\}$. In other words, $N = \mathbb{Z} \setminus \{0\}$. Let $F = \mathbb{Z} \times N$. Consider the binary relation $\sim$ on $F$ defined by the following property: $(a,b) \sim (c,d)$ is true means $ad=bc$. Prove that $\sim$ is an equivalence relation on $F$. While this question is different from the question on last week's homework, I suggest reviewing the key to HW10 (but only copy ideas that ACTUALLY still apply!)

\begin{proof}
    To show that $\sim$ is an equivalence relation on $F$, we must prove it is reflexive, symmetric, and transitive.

    \textbf{Reflexive:} Let $(a,b) \in F$. Then $ab = ba$ because multiplication is commutative. Therefore, $(a,b) \sim (a,b)$.

    \textbf{Symmetric:} Let $(a,b), (c,d) \in F$ where $(a,b) \sim (c,d)$. Then $ad = bc$. Multiplying both sides by 1 does not change the equation, so $bc = ad$. Therefore, $(c,d) \sim (a,b)$.

    \textbf{Transitive:} Let $(a,b), (c,d), (e,f) \in F$ where $(a,b) \sim (c,d)$ and $(c,d) \sim (e,f)$. Then $ad = bc$ and $cf = de$. 
    Since $b,d \in N$, they are nonzero. Multiply the first equation by $f$ and the second by $b$:
    $adf = bcf$ and $bcf = bde$
    Therefore $adf = bde$, which means $(a,b) \sim (e,f)$.

    Thus, $\sim$ is an equivalence relation on $F$.
\end{proof}

\newpage

\item 
Consider the binary relation $R$ on $\mathbb{Z}$ defined by the following property: $aRb$ is true means $10$ divides $(a-b)$. It turns out that $R$ is an equivalence relation on $\mathbb{Z}$, but I am not asking you to prove that. Instead, based on $R$ being an equivalence relation, examine the equivalence class ${[-7]}_R$. Please state THREE elements of ${[-7]}_R$, writing the notation $\in$ where this notation is appropriate. Pay careful attention to ``what kind of thingy'' belongs to ${[-7]}_R$ and ``what kind of thingy'' belongs to $R$. Briefly explain your answers. (Because I am asking you to state three elements of ${[-7]}_R$, a short answer will consist of THREE sentences that are all pretty short.)

\begin{itemize}
    \item $3 \in {[-7]}_R$ because $-7-3=-10$ which is divisible by 10. 
    \item $13 \in {[-7]}_R$ because $-7-13=-20$ which is divisible by 10. 
    \item $-17 \in {[-7]}_R$ because $-7-(-17)=10$ which is divisible by 10.
\end{itemize}

\newpage

\item 
Consider a function $f : A \to B$. (Do not pick the function yourself: I just mean that there is a function, any function, from $A$ to $B$.) As part of our set up, $H$ and $K$ are both subsets of $A$. Prove: if $H \subseteq K$, then $f(H) \subseteq f(K)$. Note: you cannot just ``$f$'' both sides: instead, look at a specific definition: what definition do you need to review for what $f(H)$ is? Also, be sure to review the definition of subset.

\begin{proof}
    Let $y \in f(H)$. Then there exists $x \in H$ such that $f(x)=y$.
    Since $H \subseteq K$, we know that $x \in K$.
    Therefore, $y=f(x)$ where $x \in K$, which means $y \in f(K)$.
    Thus, $f(H) \subseteq f(K)$.
\end{proof}

\newpage

\item 
Consider the function $f: \mathbb{R} \to \mathbb{R}$ given by the rule $f(x)=2x+3$. Prove $f$ is injective. Then prove $f$ is surjective. Hint: do not ignore quantifiers. (Be sure to apply the process to prove a ``for all'' when appropriate, the process to prove a ``there exists'' when appropriate, and the process to prove an ``if \dots then \dots'' when appropriate.)

\begin{proof}
    First, we prove $f$ is injective.
    Let $a,b \in \mathbb{R}$ and suppose $f(a)=f(b)$.
    Then $2a+3=2b+3$.
    Subtracting 3 from both sides: $2a=2b$.
    Dividing both sides by 2: $a=b$.
    Therefore, $f$ is injective.

    Next, we prove $f$ is surjective.
    Let $y \in \mathbb{R}$.
    Let $x=\frac{y-3}{2}$.
    Then $f(x)=2(\frac{y-3}{2})+3=y-3+3=y$.
    Therefore, for any $y \in \mathbb{R}$, there exists an $x \in \mathbb{R}$ such that $f(x)=y$.
    Thus, $f$ is surjective.
\end{proof}

\newpage

\item Whenever $F$ is a set of functions from $\mathbb{R}$ to $\mathbb{R}$, in this question, we define $S(F)$ to be the following:
\[ S(F) = \{x \in \mathbb{R} : \text{for all } f \in F, \text{ we have } f(x)=0\}. \]
Before moving on, try exploring what $S(F)$ actually is with $F$ being a set of cardinality $1$, then maybe with $F$ being a set of cardinality $2$ or $3$. Note that $F$ is a set, but ``what kind of thingies'' belong to this set $F$?

What's the task? Prove: if $P$ is a set of functions from  $\mathbb{R}$ to $\mathbb{R}$, and $Q$ is a set of functions from $\mathbb{R}$ to $\mathbb{R}$, and $P \subseteq Q$, then $S(Q) \subseteq S(P)$. (Advice: notice that in the definition of $S(F)$, there is a ``for all'' right after the colon. In HW10 question 2, a set (specifically, it was $B$) had a ``for all'' found inside its definition. Second advice: write what $S(Q)$ is and what $S(P)$ is: take what I gave earlier that's centered on its own line and literally replace the $F$.)

\begin{proof}
    Let $x \in S(Q)$.
    Then for all $f \in Q$, we have $f(x)=0$.
    Since $P \subseteq Q$, for any $f \in P$, we know $f \in Q$.
    Therefore, $f(x)=0$ for all $f \in P$.
    Thus, $x \in S(P)$.
    Therefore, $S(Q) \subseteq S(P)$.
\end{proof}

\newpage

\item OPTIONAL:\@ You work at the deli counter of a sandwich shop. Two separate lines form. Let's name the lines: Line A and Line B. Each line has an infinite number in it, in the following sense: for every positive integer, there is a person who is in that spot number in Line A, and also in Line B. So, we could label the fifth person in Line A using the label A5. We could label the 100th person in line B with B100. Every person in both lines wants a sandwich. You can only serve one person at a time. Describe an algorithm/process by which you can ensure that everyone will (eventually) get a sandwich. That is, person A2348769872346 may have to wait a while, but as soon as they realize the process you're following, they realize that they'll get a sandwich eventually.

Here's a process to serve everyone:
\begin{itemize}[label=$\circ$]
    \item First serve person A1, then B1
    \item Then serve person A2, then B2
    \item Then serve person A3, then B3
    \item And so on\ldots
\end{itemize}
In this way, the person at position $n$ in either line will be served no later than after $2n$ services. 
For example, person A5 knows they'll be served after at most 10 people, and B100 knows they'll be served after at most 200 people.
This ensures everyone will eventually get their sandwich.

This also means that person A2348769872346 will be served after at most $4,697,539,744,692$ people.

\end{enumerate}


\end{document}
