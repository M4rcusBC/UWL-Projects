\documentclass{article}

\usepackage[margin=0.5in]{geometry}
\usepackage{amsfonts,xcolor, amsmath, amsthm}

\newcommand\Z{\mathbb{Z}}
\newcommand\Q{\mathbb{Q}}
\newcommand\R{\mathbb{R}}
\newcommand\C{\mathbb{C}}

\begin{document}

Show your steps --- step by step. Note that when you *use* a for all, the
statement you get is slightly smaller. (So, if you *use* statement of the form
for all \ldots if \ldots then \dots the new thing you get will be of the form if \ldots
then \ldots) For the 5th question, I used a LaTeX thing called text, but to do
this, I added the amsmath package at the top.
\begin{enumerate}

    \item Prove the following by contradiction: If the Minesweeper configuration is as
          given in Figure 3.2, then Cell B is safe. (Figure 3.2 is found in the handbook
          at the top of page 69. The number 69 is printed in the corner of the page,
          though it is the 79th page of the PDF file, due to table of contents, cover
          page, etc.)

      For the sake of clarity, I will be referencing one-indexed rows and columns from Figure 3.2. For example, the top left cell is cell (1,1), and cell B is located at (8, 1).

      \begin{proof}
            Suppose that the Minesweeper configuration is as given in Figure 3.2. Also suppose for the sake of contradiction that Cell B is not safe and contains a mine. 
            Then since (8,2) has two mines adjacent to it, one being B and the other being (7, 2), then cell C must be safe.
            Because (10, 2) has three mines adjacent to it, and we know cell C is safe, then cells D and E must contain mines.
            But if both B and D contain mines, cell (9, 2) would have two mines adjacent to it instead of one, which contradicts the assumed configuration.
            Therefore, since the assumption that Cell B is not safe leads to a contradiction, Cell B must be safe.
      \end{proof}

    \newpage

    \item Prove the following by cases: If the Minesweeper configuration is as given in
          Figure 3.2, then Cell V is a mine. [Hint: start where Cells P, Q, and R are and
                  create three cases.]

      \begin{proof}
            Suppose that the Minesweeper configuration is as given in Figure 3.2. We will consider three cases based on the contents of cells P, Q, and R. 
            \begin{itemize}
                \item Case 1: If cells P, Q, and R are all safe, then since cell V has three mines adjacent to it, cell V must be a mine.
                \item Case 2: If cell P is a mine, then cell V must be a mine since it has three mines adjacent to it.
                \item Case 3: If cell Q is a mine, then cell V must be a mine since it has three mines adjacent to it.
            \end{itemize}
            Therefore, in all cases, if the Minesweeper configuration is as given in Figure 3.2, then Cell V is a mine.
      \end{proof}

    \newpage

    \item Prove: for all $a \in \mathbb{Z}$, for all $b \in \mathbb{Z}$, if $a+b$ is
          even, then $a-b$ is even.

      \begin{proof}
              Let $a, b \in \Z$ be arbitrary. Suppose $a+b$ is even. Then there exists an integer $k$ such that $a+b = 2k$. Then $a = 2k - b$. Since $2k$ and $b$ are integers, $a$ is an integer. Therefore, $a-b = 2k - b - b = 2k - 2b = 2(k-b)$. Since $k$ and $b$ are integers, $k-b$ is an integer. Therefore, $a-b$ is even.
      \end{proof}

    \newpage

    \item Prove: for all $c \in \mathbb{Z}$, we have $c$ is even if and only if $c-1$ is
          odd.

      \begin{proof}
              Let $c \in \Z$ be arbitrary. Suppose $c$ is even. Then there exists an integer $k$ such that $c = 2k$. Then $c-1 = 2k - 1 = 2(k-1) + 1$. Since $k$ is an integer, $k-1$ is an integer. Therefore, $c-1$ is odd. Now suppose $c-1$ is odd. Then there exists an integer $k$ such that $c-1 = 2k + 1$. Then $c = 2k + 1 = 2(k+1)$. Since $k$ is an integer, $k+1$ is an integer. Therefore, $c$ is even. Therefore, $c$ is even if and only if $c-1$ is odd.
      \end{proof}

    \newpage

    \item Use the hypotheses:
          \begin{itemize}
              \item H1: Every element of $D$ is an element of $A$.
              \item H2: For all $d \in D$, if $d$ raises tigers, then $d$ does not play soccer.
              \item H3: Every element of $B$ is an element of $D$.
              \item H4: Every element of $S$ is an element of $T$.
              \item H5: For all $b \in B$, if $b$ is a paramedic, then $b$ plays board games.
              \item H6: Every element of $U$ is an element of $D$.
              \item H7: For all $a \in A$, if $a$ does not raise tigers, then $a$ is a paramedic.
          \end{itemize}
          and the definitions of these sets
          \begin{itemize}
              \item $S = \{x \in D : x \text{ plays board games}\}$
              \item $T = \{c \in U : c \text{ is a barista}\}$
          \end{itemize}
          to prove: for all $b \in B$, if $b$ plays soccer, then $b$ is a barista.

          \begin{proof}
                  Let $b \in B$ be arbitrary. Suppose $b$ plays soccer. Since every element of $B$ is an element of $D$, $b \in D$. Since every element of $D$ is an element of $A$, $b \in A$. Since $b \in A$, by H7, if $b$ does not raise tigers, then $b$ is a paramedic. Since $b$ plays soccer, $b$ does not raise tigers. Therefore, $b$ is a paramedic. Since $b$ is a paramedic, by H5, $b$ plays board games. Since $b$ plays board games, $b \in S$. Since $b \in S$, $b \in T$. Since $b \in T$, $b$ is a barista. Therefore, for all $b \in B$, if $b$ plays soccer, then $b$ is a barista.
              \end{proof}

\end{enumerate}

\end{document}
