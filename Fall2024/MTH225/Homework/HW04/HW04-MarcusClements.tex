\documentclass{article}

\usepackage[margin=1.0in]{geometry}
\usepackage{amsfonts}

\newcommand\Z{\mathbb{Z}}
\newcommand\Q{\mathbb{Q}}
\newcommand\R{\mathbb{R}}
\newcommand\C{\mathbb{C}}

\begin{document}

% Answer each question on a separate page. Use the newpage command (with a backslash in front of newpage) to separate pages.
Please note: answer the first four questions in light of the rules of inference (the ``flowcharts'') we have seen. The first four questions are very quick to answer, but the point is to have you process and practice the flowcharts we have seen. (Pay attention to ``use'' versus ``prove''.)

\begin{enumerate}

\item Say we know \fbox{Olaf likes warm hugs and Anna wants to build a snowman}. What (if anything) can we conclude? Briefly explain why.



\newpage

\item Say we know \fbox{If Olaf likes warm hugs, then Anna wants to build a snowman}. In addition to this, say we also know \fbox{Olaf likes warm hugs}. What (if anything) can we conclude? Briefly explain why.



\newpage

\item Say we know \fbox{If Olaf likes warm hugs, then Anna wants to build a snowman}.  In addition to this, say we also know \fbox{Olaf does not like warm hugs}. What (if anything) can we conclude? Briefly explain why.



\newpage

\item Say $T$ is the set of all the world's turtles. Say $M(x)$ is the predicate \fbox{$x$ has $33$ feet}. What steps would we have to take to prove \fbox{There exists $h \in T [M(x)]$}?



\newpage

\item Say that $c$ and $d$ are both integers. Prove: if $c$ is even and $d$ is even, then $c-d$ is even.

\end{enumerate}


\end{document}