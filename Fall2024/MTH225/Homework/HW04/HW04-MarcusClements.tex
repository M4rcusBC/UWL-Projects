\documentclass{article}

\usepackage[margin=1.0in]{geometry}
\usepackage{amsfonts}
\usepackage{tikz}
\usepackage[normalem]{ulem}
\usepackage{amsthm}
\usepackage[english]{babel}
\usetikzlibrary{shapes.geometric, arrows}

\newcommand\Z{\mathbb{Z}}
\newcommand\Q{\mathbb{Q}}
\newcommand\R{\mathbb{R}}
\newcommand\C{\mathbb{C}}

\newtheorem{theorem}{Theorem}

\begin{document}

\definecolor{1c768acb-5c6c-5bd2-bbde-0c2f57677488}{RGB}{192, 192, 192}
\definecolor{f3551e38-74df-57e2-b793-83d7fe876c85}{RGB}{0, 0, 0}
\definecolor{0b71a967-1f15-55a5-9bb9-70efa7b4fc58}{RGB}{51, 51, 51}
\definecolor{5856d031-3da1-575c-834e-c77e9e438c62}{RGB}{162, 177, 195}

\tikzstyle{4f40dc70-f77d-567c-ad97-b7948e2d6634} = [rectangle, rounded corners, minimum width=2cm, minimum height=1cm, text centered, font=\normalsize, color=0b71a967-1f15-55a5-9bb9-70efa7b4fc58, draw=f3551e38-74df-57e2-b793-83d7fe876c85, line width=0.2, fill=1c768acb-5c6c-5bd2-bbde-0c2f57677488]
\tikzstyle{7be24b85-97d0-5b76-ba9e-d94005dca8f2} = [thick, draw=5856d031-3da1-575c-834e-c77e9e438c62, line width=2, ->, >=stealth]

% Answer each question on a separate page. Use the newpage command (with a backslash in front of newpage) to separate pages.
Please note: answer the first four questions in light of the rules of inference
(the ``flowcharts'') we have seen. The first four questions are very quick to
answer, but the point is to have you process and practice the flowcharts we
have seen. (Pay attention to ``use'' versus ``prove''.)

\begin{enumerate}

    \item Say we know \fbox{Olaf likes warm hugs \textcolor{red}{and} Anna wants to build
              a snowman}. What (if anything) can we conclude? Briefly explain why.

          \begin{itemize}
              \item Let $p$ be the statement ``Olaf like warm hugs''.
              \item Let $q$ be the statement ``Anna wants to build a snowman''.
          \end{itemize}

          We know $p \land q$ is true from the above proposition.

          From this, we can safely conclude that both $p$ and $q$ are true independently
          of each other, seen in the flowchart below.

          \begin{center}
              \begin{tikzpicture}[node distance=2cm]
                  \node (347e0db3-0031-4fa0-bd48-aa14ee508544) [4f40dc70-f77d-567c-ad97-b7948e2d6634] {$p \land q$};
                  \node (872d2943-a43d-4d7c-8214-dc609a07979e) [4f40dc70-f77d-567c-ad97-b7948e2d6634, below of=347e0db3-0031-4fa0-bd48-aa14ee508544, left of=347e0db3-0031-4fa0-bd48-aa14ee508544] {$p$};
                  \node (90348078-0426-48cb-ad8e-47e9726dd8dd) [4f40dc70-f77d-567c-ad97-b7948e2d6634, right of=872d2943-a43d-4d7c-8214-dc609a07979e, xshift=2cm] {$q$};
                  \draw [->] (347e0db3-0031-4fa0-bd48-aa14ee508544) --  (872d2943-a43d-4d7c-8214-dc609a07979e);
                  \draw [->] (347e0db3-0031-4fa0-bd48-aa14ee508544) --  (90348078-0426-48cb-ad8e-47e9726dd8dd);
              \end{tikzpicture}
          \end{center}

          \newpage

    \item Say we know \fbox{\textcolor{red}{If} Olaf likes warm hugs,
              \textcolor{red}{then} Anna wants to build a snowman}. In addition to this, say
          we also know \fbox{Olaf likes warm hugs}. What (if anything) can we conclude?
          Briefly explain why.
          \begin{itemize}
              \item Let $p$ be the statement ``Olaf like warm hugs''.
              \item Let $q$ be the statement ``Anna wants to build a snowman''.
          \end{itemize}

          We know $p \rightarrow q$ is true from the above proposition.

          From this, we can safely conclude that $q$ is true based upon $p \rightarrow q$
          and $p$ being true, seen in the flowchart below. The rule of inference used to
          derive this conclusion is \textit{Modus Ponens}.

          \begin{center}
              \begin{tikzpicture}[node distance=2cm]
                  \node (347e0db3-0031-4fa0-bd48-aa14ee508544) [4f40dc70-f77d-567c-ad97-b7948e2d6634] {$p \rightarrow q$};
                  \node (872d2943-a43d-4d7c-8214-dc609a07979e) [4f40dc70-f77d-567c-ad97-b7948e2d6634, right of=347e0db3-0031-4fa0-bd48-aa14ee508544, xshift=2cm] {$p$};
                  \node (90348078-0426-48cb-ad8e-47e9726dd8dd) [4f40dc70-f77d-567c-ad97-b7948e2d6634, below of=347e0db3-0031-4fa0-bd48-aa14ee508544, left of=872d2943-a43d-4d7c-8214-dc609a07979e] {$q$};
                  \draw [->] (347e0db3-0031-4fa0-bd48-aa14ee508544) --  (90348078-0426-48cb-ad8e-47e9726dd8dd);
                  \draw [->] (872d2943-a43d-4d7c-8214-dc609a07979e) --  (90348078-0426-48cb-ad8e-47e9726dd8dd);
              \end{tikzpicture}
          \end{center}

          \newpage

    \item Say we know \fbox{\textcolor{red}{If} Olaf likes warm hugs,
            \textcolor{red}{then} Anna wants to build a snowman}. In addition to this, say
          we also know \fbox{Olaf does \textcolor{red}{not} like warm hugs}. What (if
          anything) can we conclude? Briefly explain why.

          \begin{itemize}
              \item Let $p$ be the statement, ``Olaf likes warm hugs.''
              \item Let $q$ be the statement, ``Anna wants to build a snowman.''
              \item Naturally, we can let $\neg p$ be the statement, ``Olaf does not like warm hugs.''
          \end{itemize}

          We cannot conclude anything from the information available to us at this time.
          This is because the two propositions do not form a logically sound argument
          when used as a rule of inference. I have included the truth table which highlights my reasoning.

          \begin{center}
              \begin{tikzpicture}[node distance=2cm]
                  \node (pimpq) [4f40dc70-f77d-567c-ad97-b7948e2d6634] {$p \rightarrow q$};
                  \node (np) [4f40dc70-f77d-567c-ad97-b7948e2d6634, right of=pimpq, xshift=2cm] {$\neg p$};
                  \node (x) [4f40dc70-f77d-567c-ad97-b7948e2d6634, below of=pimpq, right of=pimpq] {We cannot conclude anything from the two above propositions.};
              \end{tikzpicture}
          \end{center}

          \rule{\linewidth}{0.5pt}

          \begin{table}[h]
              \centering
              \begin{tabular}{|c|c||c|c|}
                \hline
                $p$                        & $q$                        & $\neg p$                   & $(p \rightarrow q)$        \\ \hline
                \hline
                T  & T  & F  & T  \\ \hline
                T  & F & F & F \\ \hline
                F  & T                          & T                          & T                          \\ \hline
                F                          & F                          & T                          & T                          \\ \hline
                T  & T  & F  & T  \\ \hline
                T & F & F & F \\ \hline
                F                          & T                          & T                          & T                          \\ \hline
                F                          & F                          & T                          & T                          \\ \hline
            \end{tabular}
              \quad
              \begin{tabular}{|c|c||c|c|}
                  \hline
                  $p$                        & $q$                        & $\neg p$                   & $(p \rightarrow q)$        \\ \hline
                  \hline
                  \textcolor{red}{\sout{T}}  & \textcolor{red}{\sout{T}}  & \textcolor{red}{\sout{F}}  & \textcolor{red}{\sout{T}}  \\ \hline
                  \textcolor{blue}{\sout{T}} & \textcolor{blue}{\sout{F}} & \textcolor{blue}{\sout{F}} & \textcolor{blue}{\sout{F}} \\ \hline
                  F                          & T                          & T                          & T                          \\ \hline
                  F                          & F                          & T                          & T                          \\ \hline
                  \textcolor{red}{\sout{T}}  & \textcolor{red}{\sout{T}}  & \textcolor{red}{\sout{F}}  & \textcolor{red}{\sout{T}}  \\ \hline
                  \textcolor{blue}{\sout{T}} & \textcolor{blue}{\sout{F}} & \textcolor{blue}{\sout{F}} & \textcolor{blue}{\sout{F}} \\ \hline
                  F                          & T                          & T                          & T                          \\ \hline
                  F                          & F                          & T                          & T                          \\ \hline
              \end{tabular}
          \end{table}

          \begin{enumerate}
              \item \textcolor{blue}{Remove rows where $p \rightarrow q$ is false.}
              \item \textcolor{red}{Remove rows where $\neg p$ is false (in other words, where $p$ is true).}
          \end{enumerate}

          At this point, we are left with the following truth table:

          \begin{table}[h]
              \centering
              \begin{tabular}{|c|c||c|c|}
                  \hline
                  $p$ & $q$ & $\neg p$ & $(p \rightarrow q)$ \\ \hline
                  \hline
                  F   & T   & T        & T                   \\ \hline
                  F   & F   & T        & T                   \\ \hline
                  F   & T   & T        & T                   \\ \hline
                  F   & F   & T        & T                   \\ \hline
              \end{tabular}
          \end{table}

          From the table above, we can see that $\neg p$ is true (i.e., p is false), $q$ can be either true or false. Thus, we do not have enough information to conclude anything about $q$.

          \newpage

    \item Say $T$ is the set of all the world's turtles. Say $M(x)$ is the predicate
          \fbox{$x$ has $33$ feet}. What steps would we have to take to prove \fbox{There
              exists $h \in T [M(h)]$}?

          \begin{itemize}
              \item Let $h$ be the statement, ``$h$ is a turtle''.
              \item Let $M(h)$ be the statement, ``$h$ has $33$ feet''.
          \end{itemize}
          If we can prove that there exists at least one turtle in the set of all the world's turtles that has $33$ feet, then we can conclude that $\exists h \in T    s.t. M(h)$.

          This statement, if true, proves that there exists a turtle in the set of all
          the world's turtles that has $33$ feet.

          The rule of inference used to derive this conclusion is \textit{Existential
              Instantiation}.

          \newpage

    \item \begin{theorem}
              Say that $c$ and $d$ are both integers. Prove: if $c$ is even and $d$ is even,
              then $c-d$ is even.
          \end{theorem}

          \begin{proof}
              Let $c, d \in \mathbb{Z}$. Suppose that $c$ is even and $d$ is even. By definition,
              an integer $n$ is even if there exists an integer $k$ such that $n = 2k$. Since $c$ is
              even, there exists an integer $x$ such that $c = 2x$. Similarly, since $d$ is even,
              there exists an integer $y$ such that $d = 2y$. Substituting these values into the
              expression $c - d$, we have:

              \[c - d = 2x - 2y = 2(x - y)\]

              The value $(x - y)$ is an integer because the set of integers is closed under
              subtraction. From this, we can conclude that $c - d$ is even. Therefore, if $c$
              is even and $d$ is even, then $c - d$ is even.
          \end{proof}

\end{enumerate}

\end{document}