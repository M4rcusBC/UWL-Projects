\documentclass{article}

\usepackage[margin=1.0in]{geometry}
\usepackage{amsfonts}
\usepackage{tikz}
\usetikzlibrary{shapes.geometric, arrows}

\newcommand\Z{\mathbb{Z}}
\newcommand\Q{\mathbb{Q}}
\newcommand\R{\mathbb{R}}
\newcommand\C{\mathbb{C}}

\begin{document}

\definecolor{1c768acb-5c6c-5bd2-bbde-0c2f57677488}{RGB}{192, 192, 192}
\definecolor{f3551e38-74df-57e2-b793-83d7fe876c85}{RGB}{0, 0, 0}
\definecolor{0b71a967-1f15-55a5-9bb9-70efa7b4fc58}{RGB}{51, 51, 51}
\definecolor{5856d031-3da1-575c-834e-c77e9e438c62}{RGB}{162, 177, 195}

\tikzstyle{4f40dc70-f77d-567c-ad97-b7948e2d6634} = [rectangle, rounded corners, minimum width=2cm, minimum height=1cm, text centered, font=\normalsize, color=0b71a967-1f15-55a5-9bb9-70efa7b4fc58, draw=f3551e38-74df-57e2-b793-83d7fe876c85, line width=0.2, fill=1c768acb-5c6c-5bd2-bbde-0c2f57677488]
\tikzstyle{7be24b85-97d0-5b76-ba9e-d94005dca8f2} = [thick, draw=5856d031-3da1-575c-834e-c77e9e438c62, line width=2, ->, >=stealth]

% Answer each question on a separate page. Use the newpage command (with a backslash in front of newpage) to separate pages.
Please note: answer the first four questions in light of the rules of inference
(the ``flowcharts'') we have seen. The first four questions are very quick to
answer, but the point is to have you process and practice the flowcharts we
have seen. (Pay attention to ``use'' versus ``prove''.)

\begin{enumerate}

    \item Say we know \fbox{Olaf likes warm hugs \textcolor{red}{and} Anna wants to build
              a snowman}. What (if anything) can we conclude? Briefly explain why.

          Let $p$ be the statement ``Olaf like warm hugs'' and $q$ be the statement
          ``Anna wants to build a snowman''. We know $p \land q$ is true from the above
          proposition.

          From this, we can safely conclude that both $p$ and $q$ are true independently
          of each other, seen in the flowchart below.

          \begin{center}
              \begin{tikzpicture}[node distance=2cm]
                  \node (347e0db3-0031-4fa0-bd48-aa14ee508544) [4f40dc70-f77d-567c-ad97-b7948e2d6634] {$p \land q$};
                  \node (872d2943-a43d-4d7c-8214-dc609a07979e) [4f40dc70-f77d-567c-ad97-b7948e2d6634, below of=347e0db3-0031-4fa0-bd48-aa14ee508544, left of=347e0db3-0031-4fa0-bd48-aa14ee508544] {$p$};
                  \node (90348078-0426-48cb-ad8e-47e9726dd8dd) [4f40dc70-f77d-567c-ad97-b7948e2d6634, right of=872d2943-a43d-4d7c-8214-dc609a07979e, xshift=2cm] {$q$};
                  \node (4) [4f40dc70-f77d-567c-ad97-b7948e2d6634, below of=872d2943-a43d-4d7c-8214-dc609a07979e, right of=872d2943-a43d-4d7c-8214-dc609a07979e] {$\neg p \lor \neg q$};
                  \draw [->] (347e0db3-0031-4fa0-bd48-aa14ee508544) --  (872d2943-a43d-4d7c-8214-dc609a07979e);
                  \draw [->] (347e0db3-0031-4fa0-bd48-aa14ee508544) --  (90348078-0426-48cb-ad8e-47e9726dd8dd);
                  \draw [<->] (347e0db3-0031-4fa0-bd48-aa14ee508544) --  (4);
              \end{tikzpicture}
          \end{center}

          \newpage

    \item Say we know \fbox{\textcolor{red}{If} Olaf likes warm hugs,
              \textcolor{red}{then} Anna wants to build a snowman}. In addition to this, say
          we also know \fbox{Olaf likes warm hugs}. What (if anything) can we conclude?
          Briefly explain why.

          Let $p$ be the statement ``Olaf like warm hugs'' and $q$ be the statement
          ``Anna wants to build a snowman''. We know $p \rightarrow q$ is true from the
          above proposition.

          From this, we can safely conclude that $q$ is true based upon $p \rightarrow q$
          and $p$ being true, seen in the flowchart below. The rule of inference used to
          derive this conclusion is \textit{Modus Ponens}.

          \begin{center}
              \begin{tikzpicture}[node distance=2cm]
                  \node (347e0db3-0031-4fa0-bd48-aa14ee508544) [4f40dc70-f77d-567c-ad97-b7948e2d6634] {$p \rightarrow q$};
                  \node (872d2943-a43d-4d7c-8214-dc609a07979e) [4f40dc70-f77d-567c-ad97-b7948e2d6634, right of=347e0db3-0031-4fa0-bd48-aa14ee508544, xshift=2cm] {$p$};
                  \node (90348078-0426-48cb-ad8e-47e9726dd8dd) [4f40dc70-f77d-567c-ad97-b7948e2d6634, below of=347e0db3-0031-4fa0-bd48-aa14ee508544, left of=872d2943-a43d-4d7c-8214-dc609a07979e] {$q$};
                  \draw [->] (347e0db3-0031-4fa0-bd48-aa14ee508544) --  (90348078-0426-48cb-ad8e-47e9726dd8dd);
                  \draw [->] (872d2943-a43d-4d7c-8214-dc609a07979e) --  (90348078-0426-48cb-ad8e-47e9726dd8dd);
              \end{tikzpicture}
          \end{center}

          \newpage

    \item Say we know \fbox{\textcolor{red}{If} Olaf likes warm hugs, \textcolor{red}{then} Anna wants to build a snowman}.
          In addition to this, say we also know \fbox{Olaf does \textcolor{red}{not} like warm hugs}. What
          (if anything) can we conclude? Briefly explain why.

          \begin{center}
          \begin{tikzpicture}[node distance=2cm]
            \node (pimpq) [4f40dc70-f77d-567c-ad97-b7948e2d6634] {$p \rightarrow q$};
            \node (np) [4f40dc70-f77d-567c-ad97-b7948e2d6634, right of=pimpq, xshift=2cm] {$\neg p$};
            \node (x) [4f40dc70-f77d-567c-ad97-b7948e2d6634, below of=pimpq, right of=pimpq] {};
            \draw [->] (pimpq) --  (x);
            \draw [->] (np) --  (x);
          \end{tikzpicture}
          \end{center}

          \newpage

    \item Say $T$ is the set of all the world's turtles. Say $M(x)$ is the predicate
          \fbox{$x$ has $33$ feet}. What steps would we have to take to prove \fbox{There
              exists $h \in T [M(h)]$}?

          \newpage

    \item Say that $c$ and $d$ are both integers. Prove: if $c$ is even and $d$ is even,
          then $c-d$ is even.

\end{enumerate}

\end{document}