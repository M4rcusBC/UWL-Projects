\documentclass{article}

\usepackage[margin=1.0in]{geometry}
\usepackage{amsfonts, amsmath, amsthm, amssymb}
\usepackage{etoolbox}
\AtBeginEnvironment{align}{\setcounter{equation}{0}}

\newcommand\Z{\mathbb{Z}}
\newcommand\Q{\mathbb{Q}}
\newcommand\R{\mathbb{R}}
\newcommand\C{\mathbb{C}}

\begin{document}

% Answer each question on a separate page. Use the newpage command (with a backslash in front of newpage) to separate pages.

\begin{enumerate}

      \item Let $P(x)$ be \fbox{$x$ is a positive real number}.

            Then the statement $\exists x \in \mathbb{R} [P(x)]$ is true because there
            exists at least one positive real number.

            Then the statement $\forall x \in \mathbb{R} [P(x)]$ is false because there are
            (infinitely many) real numbers that are not positive.

            The quantifiers in front of the predicate $P(x)$ cause these statements to mean
            different things, and I believe I have accurately shown this here.

            \newpage

      \item Let $C(x)$ be \fbox{$x$ is a painter}. Let $S$ be the set of all students.
            \begin{itemize}
                  \item State $\forall x \in S [C(x)]$ in formal mathematical English.

                        ``For all painters $x$ in the set of all students $S$, $x$ is a painter.''

                  \item State $\forall x \in S [C(x)]$ in plain English. (For this part, think of how
                        you'd talk to a kid.)

                        ``Every student is a painter.''

                  \item State $\exists x \in S [C(x)]$ in formal mathematical English.

                        ``There exists a painter $x$ in the set of all students $S$ such that $x$ is a painter.''

                  \item State $\exists x \in S [C(x)]$ in plain English. (For this part, think of how
                        you'd talk to a kid.)

                        ``There is a student who is a painter.''

                  \item State $\exists x \in S [\neg C(x)]$ in plain English. (For this part, think of
                        how you'd talk to a kid.)

                        ``There is a student who is not a painter.''

            \end{itemize}

            \newpage

      \item \begin{align}
                  \forall a \in A [\forall b \in B [\exists c \in C [\forall d \in D [ & ( P(a,b) \wedge Q(c) ) \rightarrow R(d) ]]]]                                                                                    \\
                                                                                       & \equiv \neg [\forall a \in A [\forall b \in B [\exists c \in C [\forall d \in D [ ( P(a,b) \wedge Q(c) ) \rightarrow R(d) ]]]]] \\
                                                                                       & \equiv \exists a \in A \neg [\forall b \in B [\exists c \in C [\forall d \in D [ ( P(a,b) \wedge Q(c) ) \rightarrow R(d) ]]]]   \\
                                                                                       & \equiv \exists a \in A \exists b \in B \neg [\exists c \in C [\forall d \in D [ ( P(a,b) \wedge Q(c) ) \rightarrow R(d) ]]]     \\
                                                                                       & \equiv \exists a \in A \exists b \in B \forall c \in C \neg [\forall d \in D [ ( P(a,b) \wedge Q(c) ) \rightarrow R(d) ]]       \\
                                                                                       & \equiv \exists a \in A \exists b \in B \forall c \in C \exists d \in D \neg [ ( P(a,b) \wedge Q(c) ) \rightarrow R(d) ]         \\
                                                                                       & \equiv \exists a \in A \exists b \in B \forall c \in C \exists d \in D \neg [ \neg ( P(a,b) \wedge Q(c) ) \vee R(d) ]           \\
                                                                                       & \equiv \exists a \in A \exists b \in B \forall c \in C \exists d \in D [P(a,b) \wedge Q(c)] \wedge \neg R(d)
            \end{align}

            \newpage

      \item \begin{align}
                  \forall x \in \mathbb{H}, \exists y \in \mathbb{H} [P(x,y)] & \equiv \neg [\forall x \in \mathbb{H}, \exists y \in \mathbb{H} [P(x,y)]] \\
                  & \equiv \exists x \in \mathbb{H}, \neg [\exists y \in \mathbb{H} [P(x,y)]] \\
                  & \equiv \exists x \in \mathbb{H}, \forall y \in \mathbb{H} [\neg P(x,y)]
            \end{align}

            In plain English, the negation of the statement is, ``There is a human who did not send an email to any human.''

            \newpage

      \item 
      \begin{itemize}
      \item Let $A$ be the set of integers.
      \item Let $B$ be the set of all even integers.
      \item Let $P(x,y)$ be the predicate \fbox{$x$ is less than $y$} or  \fbox{$x < y$}.
      \end{itemize}
            Then, we know that $\forall a \in A [ \exists b \in B [ P(a,b) ] ]$ is true because for every integer $a$, there exists an even integer $b$ such that $a < b$.

            We also know that $\exists b \in B [ \forall a \in A [ P(a,b) ] ]$ is false because there is no even integer $b$ such that every integer $a$ is less than $b$.

\end{enumerate}

\end{document}