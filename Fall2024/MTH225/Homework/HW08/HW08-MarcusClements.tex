\documentclass{article}

\usepackage[margin=0.5in]{geometry}
\usepackage{amsfonts,xcolor, amsmath}

\newcommand\Z{\mathbb{Z}}
\newcommand\Q{\mathbb{Q}}
\newcommand\R{\mathbb{R}}
\newcommand\C{\mathbb{C}}

\begin{document}

\noindent Because this homework is due right around an exam, this is meant to be a BRIEF homework set where you address the specific skills that come up regarding set notation. Make sure your responses are informed by what we have said in class. Look to the flowcharts that we have recently drawn. Look also at the responses in {\color{blue}blue} in Essential Check 3, which serve as a study guide for Essential Check 3.

Questions 1, 2, and 4 ask about {\bf conclusions}. In those questions, state
any/all conclusions that we get from the information we start with. When there
is no conclusion, provide an explanation. (Sometimes, though not always, a way
to explain no conclusion is to say mention a specific additional thing that if
we knew it, then we could conclude something. For example, $p \rightarrow q$
cannot be used on its own, but if we also knew $p$, then we could conclude
$q$.)

\begin{enumerate}

    \item Someone defines the set $G$ for us as $G = \{u \in L : u \text{
                  rollerblades}\}$. Say that we know $k \in L$. What (if anything) can we
          conclude?

          Given the above, we cannot conclude anything further about $k$.

          If we also knew that $k$ rollerblades, then we could conclude $k \in G$.

          \newpage

    \item Someone defines the set $G$ for us as $G = \{u \in L : u \text{
                  rollerblades}\}$. Say that we know $s \in G$. What (if anything) can we
          conclude?

          From the above, we can conclude that $s \in L$ and we also conclude that $s$
          rollerblades.

          \newpage

    \item Someone defines the set $G$ for us as $G = \{u \in L : u \text{
                  rollerblades}\}$. What would we have to know/prove in order to conclude $x \in
              G$?

          If we knew that $x \in L$ and that $x$ rollerblades, then we could conclude $x
              \in G$.

          \newpage

    \item Someone defines the set $M$ for us as $M = \{\sqrt{h^2+10} \mid h \in B\}$. Say
          we know $r \in M$. What (if anything) can we conclude?

          Assuming we know $r \in M$, we can conclude that $r = \sqrt{h^2+10}$ for some
          $h \in B$.

          \newpage

    \item Someone defines the set $M$ for us as $M = \{\sqrt{h^2+10} \mid h \in B\}$.
          What would we have to know/prove in order to conclude $t \in M$?

          If we knew that $t = \sqrt{h^2+10}$ for some $h \in B$, then we could conclude
          $t \in M$.

\end{enumerate}

\end{document}
