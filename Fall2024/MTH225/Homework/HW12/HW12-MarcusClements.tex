\documentclass{article}

\usepackage[margin=0.5in]{geometry}
\usepackage{amsfonts,xcolor}
\usepackage{amsmath, amsthm} % The package amsmath will make it so that the \text{ words here } things work in the sets $S$ and $T$.

\newcommand\Z{\mathbb{Z}}
\newcommand\Q{\mathbb{Q}}
\newcommand\R{\mathbb{R}}
\newcommand\C{\mathbb{C}}

\begin{document}

\begin{enumerate}

    \item Let $f : M \to N$ and $g: N \to L$ both be surjective functions. Prove that $g
              \circ f$ is surjective. (First, look carefully at what the domain and codomain
          of $g \circ f$ are. Of course, be sure to follow the definition of surjective
          EXACTLY, and do not ignore quantifiers. Determine which ``for all'' is being
          used and which ``for all'' is being proved, and which ``there exists'' is being
          used and which ``there exists'' is being proved.)

          \begin{proof}
              To prove that $g \circ f$ is surjective, we must show that for all $\ell \in L$, there exists $m \in M$ such that $(g \circ f)(m) = \ell$.

              Let $\ell \in L$ be arbitrary. Since $g$ is surjective, there exists $n \in N$
              such that $g(n) = \ell$. Since $f$ is surjective, there exists $m \in M$ such
              that $f(m) = n$.

              Therefore, $(g \circ f)(m) = g(f(m)) = g(n) = \ell$.

              Thus, for all $\ell \in L$, there exists $m \in M$ such that $(g \circ f)(m) =
                  \ell$, proving that $g \circ f$ is surjective.
          \end{proof}

          \newpage

    \item Let $f : M \to N$ and $g: N \to L$ both be injective functions. Prove that $g
              \circ f$ is injective. (The same types of hints as the previous question, and
          in fact, at this level of math, the following advice always applies: since the
          question mentions the word injective, go review the definition first and do NOT
          ignore quantifiers. Determine which ``for all'' is being used and which ``for
          all'' is being proved.)

          \begin{proof}
              To prove that $g \circ f$ is injective, we must show that for all $m_1, m_2 \in M$, if $(g \circ f)(m_1) = (g \circ f)(m_2)$, then $m_1 = m_2$.

              Let $m_1, m_2 \in M$ be arbitrary and assume $(g \circ f)(m_1) = (g \circ
                  f)(m_2)$. Then $g(f(m_1)) = g(f(m_2))$. Since $g$ is injective, this implies
              $f(m_1) = f(m_2)$. Since $f$ is injective, this implies $m_1 = m_2$.

              Thus, for all $m_1, m_2 \in M$, if $(g \circ f)(m_1) = (g \circ f)(m_2)$, then
              $m_1 = m_2$, proving that $g \circ f$ is injective.
          \end{proof}

          \newpage

    \item Let $f : M \to N$. Prove: if $A$ and $B$ are subsets of $N$ such that $A
              \subseteq B$, then $f^{-1}(A) \subseteq f^{-1}(B)$.

          \begin{proof}
              To prove that $f^{-1}(A) \subseteq f^{-1}(B)$, we must show that for all $x \in f^{-1}(A)$, $x \in f^{-1}(B)$.

              Let $x \in f^{-1}(A)$ be arbitrary. By definition of preimage, this means $f(x)
                  \in A$. Since $A \subseteq B$, this implies $f(x) \in B$. By definition of
              preimage, this means $x \in f^{-1}(B)$.

              Thus, for all $x \in f^{-1}(A)$, $x \in f^{-1}(B)$, proving that $f^{-1}(A)
                  \subseteq f^{-1}(B)$.
          \end{proof}

          \newpage

    \item Prove that $[2,6]$ and $[11,20]$ are equicardinal. For clarifcation, both
          sets/intervals mentioned are subsets of $\mathbb{R}$.

          \newpage

          \begin{proof}
              To prove that [2,6] and [11,20] are equicardinal, we need to find a bijective function between them.

              Let $f: [2,6] \rightarrow [11,20]$ be defined by $f(x) = 3x + 5$.

              First, let's prove $f$ is injective: Let $x_1, x_2 \in [2,6]$ and assume
              $f(x_1) = f(x_2)$. Then $3x_1 + 5 = 3x_2 + 5$. Therefore $x_1 = x_2$, proving
              $f$ is injective.

              Now, let's prove f is surjective: Let $y \in [11,20]$ be arbitrary. Let $x =
                  (y-5)/3$. Then $f(x) = y$, and we need to verify $x \in [2,6]$. When y = 11, x
              = 2 When y = 20, x = 5 Since $f$ is linear and continuous, $x \in [2,6]$.
              Therefore $f(x) = y$ for some $x \in [2,6]$.

              Since $f$ is both injective and surjective, it is bijective. Therefore $[2,6]$
              and $[11,20]$ are equicardinal.
          \end{proof}

    \item Prove: if $A$ is countably infinite and $B$ is countably infinite and $C$ is
          countably infinite and $A \cap B = \emptyset$ and $A \cap C= \emptyset$ and $B
              \cap C = \emptyset$, prove $A \cup B \cup C$ is countably infinite. (Hint: it
          will be helpful to look at a past HW key where a formula for a sequence was
          given.)

          \begin{proof}
              Since A is countably infinite, there exists a bijection $f: \mathbb{N} \rightarrow A$.
              Since B is countably infinite, there exists a bijection $g: \mathbb{N} \rightarrow B$.
              Since C is countably infinite, there exists a bijection $h: \mathbb{N} \rightarrow C$.

              Define $\emptyset: \mathbb{N} \rightarrow A \cup B \cup C$ by: $\phi(n) = \begin{cases}
                      f(k) & \text{if } n = 3k-2 \text{ for some } k \in \mathbb{N} \\
                      g(k) & \text{if } n = 3k-1 \text{ for some } k \in \mathbb{N} \\
                      h(k) & \text{if } n = 3k \text{ for some } k \in \mathbb{N}
                  \end{cases}$

              To prove $\emptyset$ is injective: Let $n_1, n_2 \in \mathbb{N}$ with
              $\emptyset(n_1) = \emptyset(n_2)$. If $n_1, n_2$ came from different cases,
              they would map to different sets $(A, B, or C)$, which are disjoint. If from
              the same case, $f, g, h$ being injective implies $n_1 = n_2$. Thus $\emptyset$
              is injective.

              To prove $\emptyset$ is surjective: Let $x \in A \cup B \cup C$. If $x \in A$,
              then $x = f(k)$ for some $k$, so $x = \emptyset(3k-2)$. If $x \in B$, then $x =
                  g(k)$ for some $k$, so $x = \emptyset(3k-1)$. If $x \in C$, then $x = h(k)$ for
              some $k$, so $x = \emptyset(3k)$. Thus $\emptyset$ is surjective.

              Therefore $A \cup B \cup C$ is countably infinite.
          \end{proof}

\end{enumerate}

\end{document}
