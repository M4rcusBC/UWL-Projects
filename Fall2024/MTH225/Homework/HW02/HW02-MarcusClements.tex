\documentclass{article}

\usepackage[margin=1.0in]{geometry}
\usepackage{amsmath}

\begin{document}

\begin{enumerate}

    \item
          Find a piece-wise non-recursive formula for the sequence $a_n$ whose first
          terms are: $1,1,1,2,2,2,3,3,3,4,4,4,5,5,5$ and so on.

          Check that your formula works for $n=1$ up through $n=15$.
 
          \[
          a_n = \begin{cases}
                  1+\frac{n-1}{3}, & \text{if } n \equiv 1 \pmod{3}; \\
                  1+\frac{n-2}{3}, & \text{if } n \equiv 2 \pmod{3}; \\
                  1+\frac{n-3}{3}, & \text{if } n \equiv 0 \pmod{3}.
              \end{cases}
          \]

    \newpage

    \item Let $p$ be the proposition ``It is below freezing''. Let $q$ be the proposition
          ``It is snowing''. Write the propositions below symbolically using $p$ and $q$
          and logical operation symbols.
          \begin{itemize}
              \item It is below freezing and snowing.

                    $p \wedge q$
              \item It is below freezing but not snowing.

                    $p \wedge \neg q$
              \item It is not below freezing and it is not snowing.

                    $\neg p \wedge \neg q$
              \item If it is below freezing, it is also snowing.

                    $p \rightarrow q$
              \item It is below freezing if and only if it is not snowing.

                    $p \leftrightarrow \neg q$
          \end{itemize}

    \newpage

    \item In the parts to this question, we will examine \fbox{$r \rightarrow s$}. Before
          starting, review the truth table and the definition of implication. (Note, in
          the various parts, $r$ might not always mean the same thing, and $s$ might not
          always be the same proposition.)
          \begin{itemize}
              \item Determine the truth value of $r \rightarrow s$, given the information that $r$
                    is \fbox{Pizza grows on trees}. (In this part, we are not told what specific
                    proposition $s$ is. Answer this part based on $s$ is some mystery proposition
                    that we do not know.)

                    We can say that the truth value of $r \rightarrow s$ is always \fbox{True}
                    because $s$ is not dependant on $r$ to be true. According to the truth table
                    for this problem, this entire proposition can only be false when the leftmost
                    proposition is true and the rightmost is false. Therefore, there is no possible
                    case, given the truth value of $r$, where this proposition is false.

              \item State the converse of $r \rightarrow s$ in symbols.

                    $s \rightarrow r$
              \item State the contrapositive of $r \rightarrow s$ in symbols.

                    $\neg s \rightarrow \neg r$
              \item State the contrapositive of $r \rightarrow s$ in words, given the information
                    that $r$ is \fbox{Cats grow on trees} and $s$ is \fbox{Wisconsin is larger than
                        Alaska}.

                    If Wisconsin is not larger than Alaska, then cats do not grow on trees.

              \item Determine the truth value of the contrapositive of $r \rightarrow s$, given the
                    information that $r$ is \fbox{Cats grow on trees} and $s$ is \fbox{Wisconsin is
                        larger than Alaska}.
          \end{itemize}

          The proposition is \fbox{True}! Wisconsin is not larger than Alaska, so that
          part's good, then we evaluate $r$ and find that cats do not grow on trees! Both
          the premise and conclusion are true, which means the proposition is true!

    \newpage

    \item The following logical equivalence is called the distributive law: $p \wedge (q
              \vee r) \equiv (p \wedge q) \vee (p \wedge r)$. Apply the definitions Section
          2.3 of the Handbook of Mathematical Proof (the back half of the spiral bound)
          one step at a time. Write individual sentences: each sentence should process
          one concept and turn it into another process. After working through the
          definitions, perform a verification by writing a truth table. (For reference,
          see the discussion in the first half of Method 105.)

          \begin{itemize}

              \item We know $q \wedge r$ is true if at least one of $q$ or $r$ is true.

              \item We know $p \wedge (q \vee r)$ is true if $p$ is true and $q \vee r$ is true.

              \item We know $q \vee r$ is true if at least one of $q$ or $r$ is true.

              \item According to the distribuutive law, we can rewrite this expression as $(p
                        \wedge q) \vee (p \wedge r)$. This is true if $p \wedge q$ is true or $p \wedge
                        r$ is true.
                    \begin{table}[h]
                        \centering
                        \begin{tabular}{|c|c|c||c|c|c|c|c|}
                            \hline
                            $p$ & $q$ & $r$ & $q \vee r$ & $p \wedge (q \vee r)$ & $p \wedge q$ & $p \wedge r$ & $(p \wedge q) \vee (p \wedge r)$ \\ \hline
                            T   & T   & T   & T          & T                     & T            & T            & T                                \\ \hline
                            T   & T   & F   & T          & T                     & T            & F            & T                                \\ \hline
                            T   & F   & T   & T          & T                     & F            & T            & T                                \\ \hline
                            T   & F   & F   & F          & F                     & F            & F            & F                                \\ \hline
                            F   & T   & T   & T          & F                     & F            & F            & F                                \\ \hline
                            F   & T   & F   & T          & F                     & F            & F            & F                                \\ \hline
                            F   & F   & T   & T          & F                     & F            & F            & F                                \\ \hline
                            F   & F   & F   & F          & F                     & F            & F            & F                                \\ \hline
                        \end{tabular}
                        \caption{Since the fourth and eighth rows match, we can say these two statements are logically equivalent.}
                    \end{table}

          \end{itemize}

    \newpage

    \item Show that the proposition $\neg ( (p \wedge q) \rightarrow q)$ is a
          contradiction WITHOUT using truth tables.

          \begin{itemize}
              \item We know that $p \wedge q$ is true if both $p$ and $q$ are true.

              \item We know that $(p \wedge q) \rightarrow q$ is true if $p \wedge q$ is true and
                    $q$ is true.

              \item We know that $\neg ( (p \wedge q) \rightarrow q)$ is true if $(p \wedge q)
                        \rightarrow q$ is false.

              \item We know that $(p \wedge q) \rightarrow q$ is false if $p \wedge q$ is true and
                    $q$ is false.

              \item Since $p \wedge q$ is true if both $p$ and $q$ are true, and $q$ is false, we
                    can say that $p \wedge q$ is false.

              \item Therefore, the proposition $\neg ( (p \wedge q) \rightarrow q)$ is a
                    contradiction.

          \end{itemize}

\end{enumerate}

\end{document}

