\documentclass{article}

\usepackage[margin=1.0in]{geometry}

\begin{document}

After downloading, I recommend you rename the LaTeX right away from HW02-Instructions.tex to something like HW02-YourName.tex. After editing in Overleaf or TeXShop or TeXWorks, take the PDF (which should open in Adobe Acrobat Reader) and upload to Canvas.

Answer each question on a separate page. Use the newpage command (with a backslash in front of newpage) to separate pages. At the top of each page, you may choose to keep the question text and write your answer below, OR you can delete the question text and just have your answer. Either way is fine! (So, page 3 should have JUST the answer to question 3, OR page 3 should have the original question text followed by the answer.)

\begin{enumerate}

\item 
Find a piece-wise non-recursive formula for the sequence $a_n$ whose first terms are: $1,1,1,2,2,2,3,3,3,4,4,4,5,5,5$ and so on.

Check that your formula works for $n=1$ up through $n=15$.

\item Let $p$ be the proposition ``It is below freezing''. Let $q$ be the proposition ``It is snowing''. Write the propositions below symbolically using $p$ and $q$ and logical operation symbols.
\begin{itemize}
\item It is below freezing and snowing.
\item It is below freezing but not snowing.
\item It is not below freezing and it is not snowing.
\item If it is below freezing, it is also snowing.
\item It is below freezing if and only if it is not snowing.
\end{itemize}

\item In the parts to this question, we will examine \fbox{$r \rightarrow s$}. Before starting, review the truth table and the definition of implication. (Note, in the various parts, $r$ might not always mean the same thing, and $s$ might not always be the same proposition.)
\begin{itemize}
\item Determine the truth value of $r \rightarrow s$, given the information that $r$ is \fbox{Pizza grows on trees}. (In this part, we are not told what specific proposition $s$ is. Answer this part based on $s$ is some mystery proposition that we do not know.)
\item State the converse of $r \rightarrow s$ in symbols.
\item State the contrapositive of $r \rightarrow s$ in symbols.
\item State the contrapositive of $r \rightarrow s$ in words, given the information that $r$ is \fbox{Cats grow on trees} and $s$ is \fbox{Wisconsin is larger than Alaska}.
\item Determine the truth value of the contrapositive of $r \rightarrow s$, given the information that $r$ is \fbox{Cats grow on trees} and $s$ is \fbox{Wisconsin is larger than Alaska}.
\end{itemize}


\item The following logical equivalence is called the distributive law: $p \wedge (q \vee r) \equiv (p \wedge q) \vee (p \wedge r)$. Apply the definitions Section 2.3 of the Handbook of Mathematical Proof (the back half of the spiral bound) one step at a time. Write individual sentences: each sentence should process one concept and turn it into another process. After working through the definitions, perform a verification by writing a truth table. (For reference, see the discussion in the first half of Method 105.)

\item Show that the proposition $\neg ( (p \wedge q) \rightarrow q)$ is a contradiction WITHOUT using truth tables.

\item (OPTIONAL; 0pts) After rubbing a magic lantern, a genie appears. The genie will grant you one million dollars if you can identify the fake coin. There are $9$ coins, all except one are the same weight, the fake one is heavier than the rest. You must determine which is fake using an old fashioned balance. You may use the balance three times. (The scales are of the old balance variety. That is, a small dish hangs from each end of a rod that is balanced in the middle. The device enables you to conclude either that the contents of the dishes weigh the same or that the dish that falls lower has heavier contents than the other.) Explain how this can be done.



\end{enumerate}


\end{document}

