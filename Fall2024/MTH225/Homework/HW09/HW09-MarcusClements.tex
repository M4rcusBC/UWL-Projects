\documentclass{article}

\usepackage[margin=0.5in]{geometry}
\usepackage{amsfonts,xcolor}
\usepackage{amsmath, amsthm} % The package amsmath will make it so that the \text{ words here } things work in the sets $S$ and $T$.

\newcommand\Z{\mathbb{Z}}
\newcommand\Q{\mathbb{Q}}
\newcommand\R{\mathbb{R}}
\newcommand\C{\mathbb{C}}

\begin{document}


\begin{enumerate}

\item Let $R = \{(a,b) \in \mathbb{R} \times \mathbb{R} : 7 \text{ divides } a-b\}$. State five elements of $R$, with proper notation.

\begin{itemize}

    \item $(15, 8) \in \R$, since $7$ divides $(15-8)$.
    \item $(1.25, 8.25) \in \R$, since $7$ divides $(1.25-8.25)$.
    \item $(14, -7) \in \R$, since $7$ divides $(14-(-7))$.
    \item $(56, 7) \in \R$, since $7$ divides $(56-7)$.
    \item $(-13.15, 0.85) \in \R$, since $7$ divides $(-13.15-0.85)$.

\end{itemize}

\newpage

\item Let $R = \{(a,b) \in \mathbb{R} \times \mathbb{R} : 7 \text{ divides } a-b\}$. Note that $R$ is a binary relation on $\mathbb{R}$. Prove $R$ is reflexive.

\begin{proof}
    Want to show that for all $a \in \R$, $(a,a) \in R$. Let $a \in \R$ be arbitrary. Since $7$ divides $(a-a)$, we have $7$ divides $0$. Since $7$ divides $0$, we have $0 = 7 \cdot 0$, which is true. Therefore, $(a,a) \in R$. Since $a \in \R$ was arbitrary, we have shown that for all $a \in \R$, $(a,a) \in R$. Therefore, $R$ is reflexive.
\end{proof}

\newpage

\item Let $R = \{(a,b) \in \mathbb{R} \times \mathbb{R} : 7 \text{ divides } a-b\}$. Note that $R$ is a binary relation on $\mathbb{R}$. Prove $R$ is symmetric.

\begin{proof}
    Want to show that for all $a,b \in \R$, if $(a,b) \in R$, then $(b,a) \in R$. Let $a,b \in \R$ be arbitrary. Suppose $(a,b) \in R$. Then $7$ divides $(a-b)$. Since $7$ divides $(a-b)$, we have $a-b = 7k$ for some $k \in \Z$. Multiplying both sides by $-1$ gives $b-a = -7k$. Since $-7k \in \Z$, we have $7$ divides $(b-a)$. Therefore, $(b,a) \in R$. Since $a,b \in \R$ were arbitrary, we have shown that for all $a,b \in \R$, if $(a,b) \in R$, then $(b,a) \in R$. Therefore, $R$ is symmetric.
\end{proof}

\newpage

\item Let $R = \{(a,b) \in \mathbb{R} \times \mathbb{R} : 7 \text{ divides } a-b\}$. Note that $R$ is a binary relation on $\mathbb{R}$. Prove $R$ is transitive.

\begin{proof}
    Want to show that for all $a,b,c \in \R$, if $(a,b) \in R$ and $(b,c) \in R$, then $(a,c) \in R$. Let $a,b,c \in \R$ be arbitrary. Suppose $(a,b) \in R$ and $(b,c) \in R$. Then $7$ divides $(a-b)$ and $7$ divides $(b-c)$. Since $7$ divides $(a-b)$ and $7$ divides $(b-c)$, we have $a-b = 7k$ and $b-c = 7m$ for some $k,m \in \Z$. Adding these equations gives $a-c = 7(k+m)$. Since $k+m \in \Z$, we have $7$ divides $(a-c)$. Therefore, $(a,c) \in R$. Since $a,b,c \in \R$ were arbitrary, we have shown that for all $a,b,c \in \R$, if $(a,b) \in R$ and $(b,c) \in R$, then $(a,c) \in R$. Therefore, $R$ is transitive.
\end{proof}

\newpage

\item Based on the following set definitions:
\begin{itemize}
\item $B = \{ j\text{'s cat} : j \in T \}$
\item $L = \{ z \in V \mid z \text{ scratches furniture} \}$
\item $S = \{ m \in T : m \text{ plays hockey} \}$
\end{itemize}
and hypotheses
\begin{itemize}
\item H1: $B \subseteq L$.
\item H2: For all $c \in T$, if $c$ is a baker and $c$'s cat scratches furniture, then $c$ wins GBBO.\@
\item H3: For all $a \in T$, if $a$ does not wear a helmet, then $a$ does not skateboard.
\item H4: For all $d \in J$, if $d$ scratches furniture, then $d$ does not wear a helmet.
\item H5: $J \subseteq V$.
\end{itemize}
prove: for all $x \in S$, for all $y \in T$, if $x$ is a baker and $y$ skateboards, then $x$ wins GBBO and $y$ wears a helmet.

\begin{proof}
    Let $x \in S$ and $y \in T$ be arbitrary. Suppose $x$ is a baker and $y$ skateboards. Since $x \in S$, we know $x$ plays hockey. Since $x$ plays hockey, $x \in T$. Since $x \in T$, $x$'s cat scratches furniture. Since $x$'s cat scratches furniture, $x$ wins GBBO by hypothesis H2. Since $y$ skateboards, $y$ must wear a helmet by hypothesis H3. Therefore, for all $x \in S$, for all $y \in T$, if $x$ is a baker and $y$ skateboards, then $x$ wins GBBO and $y$ wears a helmet.
\end{proof}

\end{enumerate}

\end{document}
