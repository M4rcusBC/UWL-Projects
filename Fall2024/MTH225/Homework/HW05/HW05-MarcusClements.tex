\documentclass{article}

\usepackage[margin=0.5in]{geometry}
\usepackage{amsfonts,xcolor}
\usepackage{amsmath}
\usepackage{amsthm}
\usepackage[english]{babel}

\newcommand\Z{\mathbb{Z}}
\newcommand\Q{\mathbb{Q}}
\newcommand\R{\mathbb{R}}
\newcommand\C{\mathbb{C}}

\begin{document}

\begin{enumerate}

    \item Find a piecewise formula for the sequence $a_n$ whose first terms are\vskip1pt
          $1,1,1,1,2,2,2,2,3,3,3,3,4,4,4,4,5,5,5,5,\dots$\vskip1pt \footnote{You may not
              have a recursive formula. Your formula must work for all positive integers $n$,
              not just up to $n \leq 20$. You should certainly {\bf check} that your formula
              works for $n=5$ and for $n=18$ and so on, but your formula should also work for
              $n=2876387$. Hint: create a piecewise formula with $4$ pieces, and look at the
              key for previous HW where there was a three-piece formula.}

          The piecewise formula for this sequence can be expressed as:
          \[
              a_n = \begin{cases}
                  \frac{n + 3}{4} \text{ if } n + 3 \text{ is a multiple of } 4; \\
                  \frac{n + 2}{4} \text{ if } n + 2 \text{ is a multiple of } 4; \\
                  \frac{n + 1}{4} \text{ if } n + 1 \text{ is a multiple of } 4; \\
                  \frac{n}{4} \text{ if } n \text{ is a multiple of } 4.
              \end{cases}
          \]

          \newpage

    \item Given the hypotheses:
          \begin{itemize}
              \item H1: If Jax does not make jewlery, then Carl does not go camping. \fbox{$\neg j
                            \rightarrow \neg c$}
              \item H2: If Bo does not bake cakes, then Jax makes jewelry. \fbox {$\neg b
                            \rightarrow j$}
              \item H3: If Kofi does not fly kites, then Fiona goes fishing. \fbox{$\neg k
                            \rightarrow f$}
              \item H4: If Al does not like Archery, then Carl goes camping. \fbox{$\neg a
                            \rightarrow c$}
              \item H5: If Bo bakes cakes, then Al does not like archery. \fbox{$b \rightarrow \neg
                            a$}
              \item H6: If Carl goes camping, Kofi flies kites. \fbox{$c \rightarrow k$}
              \item H7: If Jax makes jewelry, then Matthijs records music. \fbox{$j \rightarrow m$}
              \item H8: If Kofi flies kites, then Matthijs does not record music. \fbox{$k
                            \rightarrow \neg m$}
          \end{itemize}
          Prove: If Al likes archery, then Fiona goes fishing.

          \begin{proof}
              Assume Al likes Archer (``Al does not like Archery'' is false).

              Using H5, we know that if Bo bakes cakes, then Al does not like Archery. Since
              Al likes Archery, Bo must not bake cakes.
              Then, using H2, we know that if Bo does not make cakes, then Jax must make
              jewelry. Since Bo does not bake cakes, Jax must make jewelry.
              Since Jax makes jewelrey, we can apply H7 to conclude that Matthijs records
              music.
              By H8, if Kofi flies kites, it would contradict our previous finding that
              Matthijs records music. Therefore, Kofi must not be flying kites.
              Finally, using H3, we know that if Kofi does not fly kites, then Fiona goes
              fishing. Since Kofi does not fly kites, we can conclude that Fiona goes
              fishing.

          \end{proof}

          Proof with symbols:

          \begin{itemize}
              \item Let $j$ be the statement, ``Jax makes jewelrey''.
              \item Let $b$ be the statement, ``Bo bakes cakes''.
              \item Let $k$ be the statement, ``Kofi flies kites''.
              \item Let $a$ be the statement, ``Al likes Archery''.
              \item Let $c$ be the statement, ``Carl goes camping''.
              \item Let $m$ be the statement, ``Matthijs records music''.

          \end{itemize}

          \begin{proof}
              Assume $a$ is true ($\neg a$ is false).

              Using H5, we know that if $b$ is true, then $\neg a$ is true. Since $\neg a$ is
              false, $b$ must be false ($\neg b$ must be true).
              Then, using H2, we know that if $\neg b$ is true, then $j$ is true. Since $\neg
              b$ is true, $j$ must be true.
              Since $j$ is true, we can apply H7 to conclude that $m$ is true.
              By H8, if $k$ were true, it would contradict our previous finding that $m$ is
              true. Therefore, $k$ must be false ($\neg k$ must be true).
              Finally, using H3, we know that if $\neg k$ is true, then $f$ is true. Since
              $\neg k$ is true, $f$ must be true.

          \end{proof}

          \newpage

    \item Prove: for all $a \in \mathbb{Z}$, for all $b \in \mathbb{Z}$, for all $c \in
              \mathbb{Z}$, if $a$ divides $b$, and $b$ divides $c$, then $a$ divides $c$.

              Symbolic representation: $\forall a \in \Z, \forall b \in \Z, \forall c \in \Z, (a|b \land b|c) \rightarrow a|c$.
    
          \begin{proof}
              Let $a$, $b$, and $c$ be integers. Assume $a$ divides $b$ and $b$ divides $c$. This means there exists integers $k$ and $m$ such that $b = ak$ and $c = bm$.
              Substituting $b = ak$ into $c = bm$, we get:
              \[
                c = bm = (ak)m = a(km).
              \]
              We know $km$ is an integer because the set of integers is closed under multiplication, so we can conclude that $a$ divides $c$. Therefore, if $a$ divides $b$ and $b$ divides $c$, then $a$ divides $c$.
          \end{proof}

          \newpage

    \item Prove: for all $c \in \mathbb{Z}$, for all $d \in \mathbb{Z}$, if $c$ is even
          and $d$ is odd, then $c-d$ is odd.\footnote{Note for this question: apply {\bf
                      real} algebra, so note that order of operations matters. As another note apply
              the definition of odd that is given. Do NOT make up your own ``alternate''
              definition of odd.}

          \begin{proof}
              Let $c$ and $d$ be integers. Assume $c$ is even and $d$ is odd. This means there exists integers $k$ and $m$ such that
              \[
              c = 2k \text{ and } d = 2m + 1.
            \]
              Substituting $c = 2k$ and $d = 2m + 1$ into $c - d$, we get:
              \[
              c - d = 2k - (2m + 1) = 2k - 2m - 1 = 2(k - m) - 1.
              \]
              Let $w = (k - m).$ We know that $w \in \mathbb{Z}$ because $k$ and $m$ are integers, and the set of integers is closed under subtraction. Thus, we can conclude that $c - d$ is odd. Therefore, if $c$ is even and $d$ is odd, then $c - d$ is odd.
          \end{proof}

          \newpage

    \item Using the hypotheses:
    
    I have given each of these hypotheses a name, H1 through H7, to more clearly reference them in the proof and preserve conciseness.

          \begin{itemize}
              \item For all $y \in Y$, if $y$ is a poet, then $y$ does not skateboard.
              \item For all $s \in V$, for all $t \in W$, if $s$ gives $t$ chocolate, then $t$ tap
                    dances.
              \item Every element in the set $C$ is also in the set $V$.
              \item Every element in the set $C$ is also in the set $Y$.
              \item Every element in the set $A$ is also in the set $V$.
              \item Every element in the set $B$ is also in the set $W$.
              \item For all $h \in B$, if $h$ tap dances, then $h$ is happy.
          \end{itemize}
          Prove: For all $a \in A$, for all $b \in B$, for all $c \in C$, if $c$ skateboards and $a$ gives $b$ chocolate, then $b$ is happy and $c$ is not a poet.\footnote{Take it one step at a time. For example, if we knew \fbox{For all $g \in U$, if $g$ works hard, then the boss gives $g$ a raise} and we knew \fbox{$k \in U$} was already established, then we'd get the implication \fbox{if $k$ works hard, then the boss gives $k$ a raise}. My expectation is that you write about this implication that is newly-obtained (newly-proved). Then, now that we know this implication is true, we'd be looking for an opportunity to {\bf use} it. That is, we could do something if we knew \fbox{$k$ works hard}, and we could also use the implication if we knew \fbox{the boss does not give $k$ a raise}. (My point is: slow down, take things one step at a time, and show your work. Using a for all ends up giving you a statement that is \emph{slightly} smaller.) Of course, the question that you have to work on doesn't talk about raises, working hard, etc. I just wanted to discuss based on an example that has different activities than the poetry, skateboard, tap dances, etc. in the problem.}

          \begin{proof}
              Let $a \in A$, $b \in B$, and $c \in C$. Assume $c$ skateboards and $a$ gives $b$ chocolate.
              Since every element in the set $C$ is also in the set $Y$, we conclude that $c \in y$.
              From H1, we note that if $c$ skateboards, then $c$ cannot be a poet. Thus, $c$ is not a poet.
              Since every element in the set $A$ is also in the set $V$, we have $a \in V$.
              According to H2, we know that because $a$ gives $b$ chocolate, it  follows that $b$ tap dances.
              Since every element in the set $B$ is also in the set $W$, we have $b \in W$.
              From H7, we know that since $b$ tap dances, $b$ is happy.
              Therefore, if $c$ skateboards and $a$ gives $b$ chocolate, then $b$ is happy and $c$ is not a poet.
          \end{proof}

          \newpage

    \item Optional (0pts). You walk home from a friend's house and run into another
          genie. ``Oh no, not you again!'' you exclaim. The genie lays out $12$ coins and
          a balance while saying, ``I'll give you one trillion dollars if you can
          identify which one of these twelve coins is fake, but you can only use the
          balance three times.'' You think for a minute, without even using scratch paper
          to say, ``Oh hey, that's \emph{easy} now! Give me that balance and I'll
          identify which of these coins is heaviest in no time!'' The genie grabs the
          balance from you and says, ``Not so fast! I didn't say whether the fake coin is
          lighter than the rest or heavier than the rest. Here, have some scratch
          paper.'' The genie gives an evil laugh, only to say, ``Oh, for the trillion
          dollars, you have to identify \emph{which} coin is fake, and whether the fake
          coin is heavier than a genuine coin or lighter than a genuine coin.'' Explain
          how using the balance at most three times, you can identify the fake, and
          whether it is heavier or lighter than the typical coin. Good luck!

\end{enumerate}

\end{document}