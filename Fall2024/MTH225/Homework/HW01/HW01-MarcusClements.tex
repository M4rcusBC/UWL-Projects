% Homework 01 for MTH 225 Fall 2024
% Authors: Dr. Edward Kim, Marcus Clements
\documentclass{article}

\usepackage[margin=1.0in]{geometry}
\usepackage{amsmath}

\begin{document}

\begin{center}
      \textbf{MTH 225 | Homework 01} \\
      Marcus Clements
\end{center}

I chose to denote my responses to each question with a ``$\bullet$'' to hopefully make things easier to digest. Hopefully I didn't make any mistakes with the truth table and that the notation for my responses is acceptable. Responses begin on the next page.\newline

% Once there is a percent sign, all the rest of the text on the line is ignored by LaTeX. This is called a comment. Here's where you can free-form write any notes to yourself.

% By typing % again on another line, I created another comment. Using \begin{enumerate} and \end{enumerate} creates an automatically numbered list. The next number appears with \item, and the system requires \item to appear at least once.
\begin{enumerate}

      \newpage

      \item
            Find a piece-wise non-recursive formula for the sequence $a_n$ whose first terms are: $2,2,5,5,8,8,11,11,14,$

            $14,17,17$ and so on.

            Clarification: When $x=1$ we have $y=1$. When $x=2$ then $y=2$. When $x=3$ then $y=5$. However, I am just using $x$ and $y$ for convenience. Change the $x$s to $n$s, and the $y$s to $a_n$s when you turn in your homework.

            $\bullet$ The formula for this question can be expressed as:
            \[
                  a_n = \begin{cases}
                        2 + \frac{3(n-1)}{2} \text{ for all } n \text{ if } n \text{ is odd;} \\
                        2 + \frac{3(n-2)}{2} \text{ for all } n \text{ if } n \text{ is even.}
                  \end{cases}
            \]

            \newpage

      \item
            For this homework question, let $C$ be the set of all car brands.
            \begin{enumerate}
                  \item With correct notation, state three elements of $C$.

                        $\bullet$ Let $r$ be ``Rolls-Royce'', let $f$ be ``Ford'', and let $t$ be ``Toyota''. Then $r, f, t \in C$.

                  \item Write what \fbox{$r \in C$} means in plain English.

                        $\bullet$ In plain english and applying my example for $r$, $r \in C$ means that Rolls-Royce is an element of the set of all car brands. More generally speaking, it just means that $r$ is an element of the set $C$. Therefore, we can deduce that $r$ is a car brand!

            \end{enumerate}

            \newpage

      \item
            Let $p$ be the proposition ``Madison is the capital of Wisconsin'', and let $q$ be the proposition ``Tables are a type of food'', and let $r$ be ``Dogs are not animals''. State each of the following in words:
            \begin{enumerate}
                  \item $p \wedge q$

                        $\bullet$ Madison is the capital of Wisconsin and tables are a type of food.

                  \item $p \vee q$

                        $\bullet$ Madison is the capital of Wisconsin or tables are a type of food.

                  \item $(r \wedge p) \rightarrow q$

                        $\bullet$ If dogs are not animals and Madison is the capital of Wisconsin, then tables are a type of food.

                  \item $q \rightarrow \neg r$

                        $\bullet$ If tables are a type of food, then dogs are animals.

                  \item $r \wedge r$

                        $\bullet$ Dogs are not animals and dogs are not animals.

            \end{enumerate}

            \newpage

      \item
            Let $p$ be the proposition ``Madison is the capital of Wisconsin'', and let $q$ be the proposition ``Tables are a type of food'', and let $r$ be ``Dogs are not animals''. For each proposition below, state in symbols and ALSO determine if the proposition is true or false (and provide some explanation for why).
            \begin{enumerate}
                  \item Madison is the capital of Wisconsin and dogs are not animals.

                        $\bullet$ Symbolic representation: $p \wedge r$

                        $\bullet$ Truth value: False

                        $\bullet$ Explanation: We know conjunction of two propositions is true, in this case, only if both $p$ and $r$ are true. Since $r$ (``Dogs are not animals'') is clearly false, the entire statement is false.

                  \item Madison is the capital of Wisconsin or dogs are animals.

                        $\bullet$ Symbolic representation: $p \vee \neg r$

                        $\bullet$ Truth value: True

                        $\bullet$ Explanation: We also know the disjunction of two propositions is true if either $p$ or $\neg r$ (the negation of $r$) is true. Since p (``Madison is the capital of Wisconsin'') is true, the entire statement is true, regardless of the truth value of $\neg r$.

                  \item If dogs are animals, then tables are a type of food.

                        $\bullet$ Symbolic representation: $\neg r \rightarrow q$

                        $\bullet$ Truth value: True

                        $\bullet$ Explanation: This is a conditional statement, which is true unless the antecedent ($\neg r$) is true and the consequent ($q$) is false. Since $\neg r$ (``Dogs are not animals'') is false, the entire statement is true, regardless of the truth value of $q$.

                  \item If Madison is not the capital of Wisconsin, then dogs are not animals.

                        $\bullet$ Symbolic representation: $\neg p \rightarrow r$

                        $\bullet$ Truth value: False

                        $\bullet$ Explanation: This is also a conditional statement. It's true unless the antecedent ($\neg p$) is true and the consequent ($r$) is false. Since both $\neg p$ (``Madison is not the capital of Wisconsin'') and $r$ (``Dogs are not animals'') are false, the entire statement is false.

                  \item Dogs are not animals or dogs are animals.

                        $\bullet$ Symbolic representation: $r \vee \neg r$

                        $\bullet$ Truth value: True

                        $\bullet$ Explanation: This statement is a tautology, meaning it is always true regardless of the truth values of $r$ and $\neg r$. This is because this equation represents a disjunction (OR) of a statement and its negation. At least one of these must be true, therefore the statement is always true.

            \end{enumerate}

            \newpage

      \item Provide a complete truth table for $(p \rightarrow \neg q) \wedge (q \vee r)$.

            Clarification: for a complete truth table, I am expecting you to ``show your work'' by giving relevant columns that lead up to the final column. For example, one of the columns you should display work for is $q \vee r$.

            \begin{table}[h]
                  \centering
                  \begin{tabular}{|c|c|c||c|c|c|c|}
                        \hline
                        $p$ & $q$ & $r$ & $\neg q$ & $(p \rightarrow \neg q)$ & $(q \vee r)$ & $(p \rightarrow \neg q) \wedge (q \vee r)$ \\ \hline
                        T   & T   & T   & F        & F                        & T            & F                                          \\ \hline
                        T   & T   & F   & F        & F                        & T            & F                                          \\ \hline
                        T   & F   & T   & T        & T                        & T            & T                                          \\ \hline
                        T   & F   & F   & T        & T                        & F            & F                                          \\ \hline
                        F   & T   & T   & F        & T                        & T            & T                                          \\ \hline
                        F   & T   & F   & F        & T                        & T            & T                                          \\ \hline
                        F   & F   & T   & T        & T                        & T            & T                                          \\ \hline
                        F   & F   & F   & T        & T                        & F            & F                                          \\ \hline
                  \end{tabular}
                  \caption{Truth Table for $(p \rightarrow \neg q) \wedge (q \vee r)$}
            \end{table}

\end{enumerate}

\end{document}