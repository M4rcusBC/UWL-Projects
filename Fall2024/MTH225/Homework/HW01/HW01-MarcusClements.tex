\documentclass{article}

\usepackage[margin=1.0in]{geometry}

\begin{document}

It is easiest to download the instructions file, since you can see exactly how I got certain symbols. My suggestion is to rename the file right away, so that you don't actually turn in the instructions, but instead you turn in your work. If your name is Peter Parker, perhaps a good name for your LaTeX file is HW01-PeterParker.tex, and then this would end up created a PDF file called HW01-PeterParker.pdf. I am not requiring any specific naming, but renaming the instructions file right away is helpful.

Answer each question on a separate page. Use the newpage command (with a backslash in front of newpage) to separate pages. At the top of each page, you may choose to keep the question text and write your answer below, OR you can delete the question text and just have your answer. Either way is fine!

% Once there is a percent sign, all the rest of the text on the line is ignored by LaTeX. This is called a comment. Here's where you can free-form write any notes to yourself.

% By typing % again on another line, I created another comment. Using \begin{enumerate} and \end{enumerate} creates an automatically numbered list. The next number appears with \item, and the system requires \item to appear at least once.
\begin{enumerate}

\item 
Find a piece-wise non-recursive formula for the sequence $a_n$ whose first terms are: $2,2,5,5,8,8,11,11,14,14,17,17$ and so on.

Clarification: When $x=1$ we have $y=1$. When $x=2$ then $y=2$. When $x=3$ then $y=5$. However, I am just using $x$ and $y$ for convenience. Change the $x$s to $n$s, and the $y$s to $a_n$s when you turn in your homework.

\item 
For this homework question, let $C$ be the set of all car brands.
\begin{enumerate}
\item With correct notation, state three elements of $C$.
\item Write what \fbox{$r \in C$} means in plain English.
\end{enumerate}

\item 
Let $p$ be the proposition ``Madison is the capital of Wisconsin'', and let $q$ be the proposition ``Tables are a type of food'', and let $r$ be ``Dogs are not animals''. State each of the following in words:
\begin{enumerate}
\item $p \wedge q$
\item $p \vee q$
\item $(r \wedge p) \rightarrow q$
\item $q \rightarrow \neg r$
\item $r \wedge r$
\end{enumerate}

\item 
Let $p$ be the proposition ``Madison is the capital of Wisconsin'', and let $q$ be the proposition ``Tables are a type of food'', and let $r$ be ``Dogs are not animals''. For each proposition below, state in symbols and ALSO determine if the proposition is true or false (and provide some explanation for why).
\begin{enumerate}
\item Madison is the capital of Wisconsin and dogs are not animals.
\item Madison is the capital of Wisconsin or dogs are animals.
\item If dogs are animals, then tables are a type of food.
\item If Madison is not the capital of Wisconsin, then dogs are not animals.
\item Dogs are not animals or dogs are animals.
\end{enumerate}

\item Provide a complete truth table for $(p \rightarrow \neg q) \wedge (q \vee r)$.

Clarification: for a complete truth table, I am expecting you to ``show your work'' by giving relevant columns that lead up to the final column. For example, one of the columns you should display work for is $q \vee r$.

\end{enumerate}


\end{document}